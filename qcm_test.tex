\documentclass[12pt,a4paper]{article}
\usepackage[utf8]{inputenc}
\usepackage[french]{babel}
\usepackage{amsmath}
\usepackage{amssymb}
\usepackage{amsthm}
\usepackage{enumitem}

\title{QCM de Mathématiques - Test de Formules LaTeX}
\author{Test}
\date{\today}

\begin{document}

\maketitle

\section{Questions à Choix Multiples}

\subsection{Question 1 - Équations du second degré}

Soit l'équation : $ax^2 + bx + c = 0$ avec $a \neq 0$

Les solutions de cette équation sont données par :
\begin{equation}
x = \frac{-b \pm \sqrt{b^2 - 4ac}}{2a}
\end{equation}

Le discriminant est : $\Delta = b^2 - 4ac$

\textbf{Question :} Si $\Delta > 0$, combien de solutions réelles l'équation possède-t-elle ?

\begin{enumerate}[label=\alph*)]
    \item 0 solution
    \item 1 solution
    \item 2 solutions distinctes
    \item Une infinité de solutions
\end{enumerate}

\textbf{Réponse correcte : c)}

\subsection{Question 2 - Dérivées}

Soit la fonction $f(x) = x^n$ où $n \in \mathbb{N}$

\textbf{Question :} Quelle est la dérivée de $f$ ?

\begin{enumerate}[label=\alph*)]
    \item $f'(x) = nx^{n-1}$
    \item $f'(x) = nx^{n}$
    \item $f'(x) = (n-1)x^{n-1}$
    \item $f'(x) = x^{n-1}$
\end{enumerate}

\textbf{Réponse correcte : a)}

\subsection{Question 3 - Intégrales}

\textbf{Question :} Quelle est la valeur de $\displaystyle\int_0^1 x^2 \, dx$ ?

\begin{enumerate}[label=\alph*)]
    \item $\dfrac{1}{2}$
    \item $\dfrac{1}{3}$
    \item $\dfrac{2}{3}$
    \item $1$
\end{enumerate}

Calcul :
\begin{align}
\int_0^1 x^2 \, dx &= \left[\frac{x^3}{3}\right]_0^1 \\
&= \frac{1^3}{3} - \frac{0^3}{3} \\
&= \frac{1}{3}
\end{align}

\textbf{Réponse correcte : b)}

\subsection{Question 4 - Limites}

\textbf{Question :} Quelle est la valeur de $\displaystyle\lim_{x \to 0} \frac{\sin(x)}{x}$ ?

\begin{enumerate}[label=\alph*)]
    \item $0$
    \item $1$
    \item $+\infty$
    \item La limite n'existe pas
\end{enumerate}

\textbf{Réponse correcte : b)}

\subsection{Question 5 - Matrices}

Soit les matrices :
\[
A = \begin{pmatrix}
1 & 2 \\
3 & 4
\end{pmatrix}
\quad \text{et} \quad
B = \begin{pmatrix}
5 & 6 \\
7 & 8
\end{pmatrix}
\]

\textbf{Question :} Quelle est la valeur de $A \times B$ ?

\begin{enumerate}[label=\alph*)]
    \item $\begin{pmatrix} 19 & 22 \\ 43 & 50 \end{pmatrix}$
    \item $\begin{pmatrix} 5 & 12 \\ 21 & 32 \end{pmatrix}$
    \item $\begin{pmatrix} 6 & 8 \\ 10 & 12 \end{pmatrix}$
    \item $\begin{pmatrix} 19 & 43 \\ 22 & 50 \end{pmatrix}$
\end{enumerate}

Calcul :
\begin{align}
A \times B &= \begin{pmatrix}
1 \times 5 + 2 \times 7 & 1 \times 6 + 2 \times 8 \\
3 \times 5 + 4 \times 7 & 3 \times 6 + 4 \times 8
\end{pmatrix} \\
&= \begin{pmatrix}
19 & 22 \\
43 & 50
\end{pmatrix}
\end{align}

\textbf{Réponse correcte : a)}

\subsection{Question 6 - Séries}

\textbf{Question :} Quelle est la valeur de $\displaystyle\sum_{n=1}^{\infty} \frac{1}{2^n}$ ?

\begin{enumerate}[label=\alph*)]
    \item $\dfrac{1}{2}$
    \item $1$
    \item $2$
    \item La série diverge
\end{enumerate}

C'est une série géométrique avec $a = \frac{1}{2}$ et $r = \frac{1}{2}$ :
\[
\sum_{n=1}^{\infty} \frac{1}{2^n} = \frac{a}{1-r} = \frac{\frac{1}{2}}{1-\frac{1}{2}} = \frac{\frac{1}{2}}{\frac{1}{2}} = 1
\]

\textbf{Réponse correcte : b)}

\subsection{Question 7 - Probabilités}

\textbf{Question :} On lance un dé équilibré à 6 faces. Quelle est la probabilité d'obtenir un nombre pair ?

\begin{enumerate}[label=\alph*)]
    \item $\dfrac{1}{6}$
    \item $\dfrac{1}{3}$
    \item $\dfrac{1}{2}$
    \item $\dfrac{2}{3}$
\end{enumerate}

Les nombres pairs sont : $\{2, 4, 6\}$, soit 3 cas favorables sur 6 possibles.
\[
P(\text{pair}) = \frac{3}{6} = \frac{1}{2}
\]

\textbf{Réponse correcte : c)}

\subsection{Question 8 - Trigonométrie}

\textbf{Question :} Quelle est la valeur de $\cos^2(x) + \sin^2(x)$ pour tout $x \in \mathbb{R}$ ?

\begin{enumerate}[label=\alph*)]
    \item $0$
    \item $1$
    \item $2$
    \item Dépend de $x$
\end{enumerate}

\textbf{Réponse correcte : b)} (Identité trigonométrique fondamentale)

\subsection{Question 9 - Logarithmes}

\textbf{Question :} Quelle est la valeur de $\ln(e^3)$ ?

\begin{enumerate}[label=\alph*)]
    \item $e^3$
    \item $3$
    \item $3e$
    \item $\dfrac{1}{3}$
\end{enumerate}

Propriété : $\ln(e^x) = x$, donc $\ln(e^3) = 3$

\textbf{Réponse correcte : b)}

\subsection{Question 10 - Vecteurs}

Soient les vecteurs $\vec{u} = \begin{pmatrix} 1 \\ 2 \end{pmatrix}$ et $\vec{v} = \begin{pmatrix} 3 \\ 4 \end{pmatrix}$

\textbf{Question :} Quel est le produit scalaire $\vec{u} \cdot \vec{v}$ ?

\begin{enumerate}[label=\alph*)]
    \item $5$
    \item $7$
    \item $11$
    \item $14$
\end{enumerate}

Calcul :
\[
\vec{u} \cdot \vec{v} = 1 \times 3 + 2 \times 4 = 3 + 8 = 11
\]

\textbf{Réponse correcte : c)}

\end{document}
